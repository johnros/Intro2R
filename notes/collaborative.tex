\chapter{Recommender Systems and Collaborative Filtering}
\label{sec:collaborative_filtering}


% % % % recommender systems % % % % % % %
\section{Recommender Systems}

A recommender system is a software that, as the name suggests, gives recommendations to the user. 
Notable examples include Book recommendations by Amazon, and film recommendations by Netflix.
The two main approaches to recommender systems include \emph{content filtering} and \emph{collaborative filtering}.



% % % % Content filtering % % % % % %
\section{Content Filtering}

In content filtering, the system is assume to have some backround information on the user (say, because he logged in), and uses
this information to give him recommendations.
The recommendation in this case, is approached as a supervised learning problem: 
the system learns to predict a product's rating ($y$) based on the user's features ($x$).
It then computes the rating for many candidate products and recommends a set with high predicted ratings.



% % % % colaborative filtering % % % % %
\section{Collaborative Filtering}

Unlike content filtering, in \emph{collaborative filtering}, there is no external information on the user or the products, besides the ratings of other users. 
The term collaborative filtering, was coined by the authors of the first such system-- Tapestry \citep{goldberg_using_1992}. 

Collaborative filtering can be approached as a supervised learning problem, or as an unsupervised learning problem. This is because it is neither. 
It is essentially a \emph{missing data} problem.\marginnote{Missing Data}
To see this consider a matrix of rankings, $\rankings$ where the $i,j$'th entry, $\ranking_{i,j}$, is the ranking of user $i$ movie $j$. 
Predicting $\ranking_{i,j'}$, i.e., the ranking of a currently unseen movie, is essentially an imputation of a missing value.


The two main approaches to collaborative filtering include \emph{neighbourhood methods}, and \emph{latent factor models} \cite{koren_matrix_2009}.

\subsubsection{Neighbourhood Methods}
The neighbourhood methods to collaborative filtering rest on the assumption that similar individuals have similar tastes. 
If someone similar to individual $i$ has seen movie $j'$, then $i$ should have a similar opinion.

The notion of using the neighbourhood of a data point is not a new one. We have seen it being used for supervised learning in kernel regression (\S\ref{sec:kernel}) and KNN (\S\ref{sec:knn}).

Neighbourhood methods for collaborative filtering, or missing data imputation in general, can thus be seen as supervised learning problems, and solved in the same way.


\begin{remark}[Collaborative Filtering and Other Supervised Learning Methods]
If you are wondering, why only neighbourhood methods for supervised learning apply to collaborative filtering, you are right.
Any supervised learning method can be applied to impute entries in $\rankings$. Neighbourhood 
methods are merely the most popular.
\end{remark}


\subsubsection{Latent Factor Models}
The latent factor approach to collaborative filtering rests on the assumption that the rankings are a function of some latent user attributes and latent movie attributes. 
This idea is not a new one, as we have seen it in the context of unsupervised learning in factor analysis (FA) (\S\ref{sec:factor_analysis}), independent component analysis (ICA) (\S\ref{sec:ica}), and other latent space generative models.
From the relation between FA and PCA, it should also come as no surprise that collaborative filtering can also be viewed as a matrix approximation problem. 
When approximating the rankings matrix ($\rankings$) with some simple manifold $\manifold$, en-passant, we also impute the missing entries in $\rankings$.

We now see that collaborative filtering, and missing data imputation in general, can also be approached as an unsupervised learning problem. 



